\documentclass[12pt,letterpaper]{scrreprt}

%----------------------------------------------------------------------
%				Required Packages
%----------------------------------------------------------------------
\usepackage{usecases}
\usepackage{enumitem}
%\usepackage{paralist}
%\usepackage{tabto}
\usepackage[top=2cm,bottom=3.5cm]{geometry}

%----------------------------------------------------------------------
%				Title Page
%----------------------------------------------------------------------
\title{CS383: Software Engineering}
\subtitle{HW2: Use Cases\\Fall 2013}
\author{CS383 Python Team} % Include everyone's name?
\date{}

%----------------------------------------------------------------------
%				Additional Settings
%----------------------------------------------------------------------
\setcounter{tocdepth}{3}
\setcounter{secnumdepth}{3}




%----------------------------------------------------------------------
%				Use Case Document Start
%----------------------------------------------------------------------
\begin{document}

\maketitle

\chapter{Use cases}
        \section{Movement}
                \subsection{Use Case - Detection Routine}
                \begin{usecase}
                  \addtitle{Movement III}{Detection Routine}
                  \addfield{Summary}{Used to detect a ship in enemy space}
                  \additemizedfield{Actors}{
                    \item Character Spaceship
					\item Enemy PDB
                  }
                  \additemizedfield{Preconditions}{
                    \item Enemy PDB is on planet
                  }
                  \additemizedfield{Evasion Value}{
                    \item Determine Evasion Value of character's ship.
                  \additemizedfield{Detection Results}{
                    \item Using Detection Table, determine Detection Status:
                      \begin{enumerate}
                        \item Undetected
                        \item Detected
						\item Detected and Damaged
						\item Eliminated
                      \end{enumerate}
                    \item If ship is undetected, proceed with movement. If characters are detected, then they become undetected.
					\item If ship is detected, then change character counters to detected side. Do not flip over spaceship, proceed with movement.
					\item If ship is detected and damaged, flip both characters and ship counters, proceed with movement. Depending on scale of game, spaceship may be destroyed.
					\item If ship is eliminated, all characters aboard are destroyed (removed from play).
				  }
				  \additemizedfield{Enemy PDB Level}{
				    \item If Enemy PDB level is 0, then ship can not be detected and damaged or eliminated, only detected.
					\item If Enemy PDB is Down, no detection (or damaged or detection or eliminated) is possible.
				  }
				  \additemizedfield{Multiple Detection Routines}{
				    \item A spaceship can go through two detection routines in a turn:
					  \begin{enumerate}
                        \item Leaving a planet.
                        \item Arriving on a planet.
                      \end{enumerate}
					\item When going from Environ to Environ, the Detection routine only occurs once.
				  }
				  \additemizedfield{Military Units}{
				    \item Military Units are always considered detected. They never go through the Detection Routine.
					\item PDB could attack the military units.
				  }
				  \additemizedfield{Characters without a ship}{
				    \item Characters without a ship moving from Environ to Environ are no longer detected.
					\item This may not occur if detected characters already in the Environ being moved to.
				  }
				  \additemizedfield{Damaged Spaceships}{
					\item If a spaceship that is already damaged receives a Damage Result, that result is treated as an Eliminated.
					\item This only occurs if a spaceship is damaged as a result of Multiple Detection Routines (Damaged on leaving, and damaged on arrival).
				  }
				\end{usecase}
                \subsection{Use Case - Effects of Detection}
                \begin{usecase}
                  \addtitle{Movement IV}{Effects of Detection}
                  \addfield{Summary}{When all characters in an environ are considered detected.}
                  \additemizedfield{Actors}{
                        \item Rebel Player
                        \item Imperial Player
                  }
                  \additemizedfield{Preconditions}{
                        \item All characters in an environ must be detected.
                  }
                  \additemizedfield{Reaction Move}{
                        \item Enemy is may make a reaction move if there are detected characters in an Environ.
						\item Enemy may conduct a search for those characters during the Search Phase.
                  }
                  \additemizedfield{Action Cards}{
                        \item Characters may be detected as a result of Action Cards when resolving missions.
						\item Characters may be detected by conducting a search.
                  }
				  \additemizedfield{Detected Characters and the Detection Routine}{
						\item Detected Characters in a detection routine undergo a column shift adjustement on the Detection Table.
				  }
				  \additemizedfield{Assasination Missions}{
						\item Assassination missions may be performed only against a detected Enemy Character.
				  }
				  \additemizedfield{Spaceship leaving planet}{
						\item When a friendly spaceship leaves a planet when an Enemy Player is ineligible to conduct the Detection Routine, the spaceship is no longer detected.
				  }
                \end{usecase}
                \subsection{Use Case - Stacking}
                \begin{usecase}
                  \addtitle{Movement V}{Stacks}
                  \addfield{Summary}{A player may have two types of Stacks in a Environ: Military Units and characters eligible to perform missions.}
                  \additemizedfield{Actors}{
                    \item Imperial Player
                    \item Rebel Player
                  }
                  \additemizedfield{Preconditions}{
                    \item Environ Size Number can not be exceeded for Military Units.
                  }
                  \additemizedfield{Military Unit Limits}{
                    \item The Environ Size Number is per Player.
                    \item When the Military Unit Stack Size exceeds the Environ Size Number, the player must choose which military units to eliminate.
					\item Characters and their spaceships never count toward this limit.
                  }
                  \additemizedfield{Stacks with Military Units}{
                    \item A stack with any military units is moved as per moving military units.
					\item It undergoes attacks by Enemey PDBs.
					\item It takes part in Military Combats if called for.
					\item These stacks are always considered detected.
                  }
                  \additemizedfield{Leader of Military Stack}{
                    \item Characters with a Leadership Rating of 1 or more may be declared the leader of a military stack whenever the owning Player wishes.
                  }
				  \additemizedfield{Non-Military Stack}{
					\item A stack with no military units, each character and spaceship is moved seperately.
					\item Each ship undergoes the Detection Routine seperately.
				  }
				  \additemizedfield{Moving Spaceships}{
					\item A spaceship can only be moved when stacked with:
					  \begin{enumerate}
                        \item Military units.
                        \item Characters with a Navigation Rating of 1 or more. 
                      \end{enumerate}
				  }
				  \additemizedfield{Moving Characters}{
					\item A character may only be moved from Environ to Environ on another planet when stacked with:
					  \begin{enumerate}
                        \item a Spaceship.
                        \item Mobile Military Units.
                      \end{enumerate}
					\item Spaceships have a maximum number of characters that can be stacked with them.
				  }
				  \additemizedfield{End of Movement Segment}{
					\item When finished moving stacks, the player's units should be organized in two stacks in each Environ:
					  \begin{enumerate}
                        \item One containing Military Units
                        \item One containing characters and spaceships who will be performing missions.
                      \end{enumerate}
					\item Characters that were moved with military units don't have to be stacked with them at the end of the movement segment.
					\item Characters stacked with Military units during Mission Phase may not perform missions.
				  }
                \end{usecase}
        \subsection{Use Case - Enemy Reaction Moves}
                \begin{usecase}
                  \addtitle{Movement VI}{Enemy Reaction Moves}
                  \addfield{Summary}{A player may The Non-Phasing player may make a reaction move for each Environ that contains a military unit or detected character that belongs to the Phasing Player.}
                  \additemizedfield{Actors}{
                    \item Imperial Player
                    \item Rebel Player
                  }
                  \additemizedfield{Preconditions}{
                    \item Phasing Player has detected and/or military units on a planet.
					\item Non-Phasing Player has a character, military unit, or a military unit and a leader on the same planet.
                  }
                  \additemizedfield{Reaction Move Limitations}{
                    \item A Reaction Move may not be made from one planet to another.
					\item A Reaction Move may be made from one Environ to another Environ on the same planet.
                  }
                  \additemizedfield{Reaction Move Stacking}{
                    \item A unit moved as a Reaction Move may be stacked with any of the owning Player's units currently in the Environ (While following all the rules of Stacking).
					\item This unit may take part in all succeeding activities in that Environ this Game-Turn.
                  }
                \end{usecase}
        \subsection{Use Case - Detection Table}
                \begin{usecase}
                  \addtitle{Movement VII}{Detection Table}
                  \addfield{Summary}{Table for use in the Movement Rules}
                  \additemizedfield{Actors}{
                    \item None
                  }
                  \additemizedfield{Preconditions}{
                    \item A Player must be in Movement Phase.
				  }
                  \additemizedfield{Detection Table Implementation}{
                    \item Detection Table must be implemented in the game.
					\item Table should be implemented as a two-dimensional list or array.
				  }
                \end{usecase}
\end{document}
